%!TEX root = paper.tex

\section{Discussion}
\label{sec:discuss}

% In theory power of DFs lie in their flexibility. Can be used to address a wide range of tasks, integrating variates of different nature and designing expressive generic or task-specific features. Can be coupled with a number of other techniques, used as part of a more global framework (\eg atlas forests). In practice this would be offset by a number of technicalities in DF training (-> cumbersome, unreliable). We turn this from partly unrealized potential into a potent strength via a number of contributions that make the approach more principled and empirically sound. fast experimentation, minutes or hrs

This is where we mention the BRATS 2016 challenge. The interesting part is to discuss numbers found in the literature on the 2015 training set (scores in the 90's) and the outcome at the BRATS challenge. We even have the scores for the 2016 test set, very interesting to discuss it.

Also, recall that many application-specific improvements to the DF framework have been proposed in the past years (or multi-stage procedures which DFs are only a part of) and they would also benefit us here (\eg segmenting WM/GM/CSF in brain applications). The standpoint of the paper is complementary, proposing generic, efficient, widely applicable improvements to DF training that will improve accuracy in most settings.

Might be worth pointing out once more the fast experimentation / practical aspect of the approach. Compare training times to the literature here (keep the results section as clean, concise and factual as possible). 