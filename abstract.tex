%!TEX root = paper.tex

We revisit Auto-Context Forests for brain tumour segmentation in multi-channel magnetic resonance images. 
Semantic context is progressively built and refined via successive layers of Decision Forests (DFs). 
%This sequential approach allows to decompose complex segmentation tasks into a series of simpler subtasks, \eg one that encodes the hierarchical structure between labels. 
Specifically, we make the following contributions: 1) \textit{improved generalization} via an efficient 
node-splitting criterion based on hold-out estimates,
%of generalization to enhance the discriminative power of nodes in decision ensembles. 
2) \textit{increased compactness} at a tree-level,
%we derive a principled, assumption-free trade-off between data-fit and model complexity, 
thereby yielding shallow discriminative ensembles trained orders of magnitude faster, and 3) 
\textit{guided semantic bagging} that exposes latent data-space semantics captured by forest pathways.
%and specialize subsequent classifiers over the resulting semantic regions to boost accuracy. 
The proposed framework is practical: the per-layer training is fast, modular and robust. It was a top performer in the MICCAI 2016 BRATS (\uline{Bra}in \uline{T}umour \uline{S}egmentation) challenge, and this paper aims to discuss and provide details about the challenge entry.
